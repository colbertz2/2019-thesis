\thispagestyle{plain}
\begin{center}
    % {\Large\textbf{Development of a Visible-NIR Photoluminescence Microspectrometer}}
 
    % \vspace{0.4cm}
    % by \\
    % \textbf{Zachary Colbert}
 
    % \vspace{0.9cm}
    \textbf{Abstract}
\end{center}

Photoluminescence (PL) is the process by which light is absorbed and re-emitted by a material. In solid-state physics, PL is an important characteristic measurement for studying the electron band structure of molecules. Here we report the design and construction of a simple instrument which can accurately measure PL in thin-film materials on a micron spatial scale. We accomplish this by coupling a diode laser system to a metallurgical microscope, using optical filters to block reflected light. The instrument is equipped with a digital camera for imaging, and a compact spectrometer for measuring fluorescence spectra. We demonstrate the use of the instrument on thin-film samples of crystalline anthradithiophene and cadmium selenide quantum dots.

% FEEDBACK FROM CLASS: Very readable, terms understandable. Include more details about
% results: how do we determine that the measurements are good?
% Add future work notes