\thispagestyle{plain}
\begin{center}
    % {\Large\textbf{Development of a Visible-NIR Photoluminescence Microspectrometer}}
 
    % \vspace{0.4cm}
    % by \\
    % \textbf{Zachary Colbert}
 
    % \vspace{0.9cm}
    \textbf{Abstract}
\end{center}

Photoluminescence (PL) emission is an important characteristic measurement in the field of solid-state physics research. It reveals information about the electron band structure of molecules, and is particularly useful in studying thin-film materials on micron spatial scales. This work sought to design and build a simple instrument which can accurately measure PL in thin-film materials by applying targeted, monochromatic illumination to the sample.

By coupling a diode laser system into a metallurgical microscope, we were able to apply sufficiently energetic, monochromatic light to a small area and achieve sample excitation. Using optical filters to block reflected laser light, we collect emissions with a commercial spectrometer. The resulting PL emission spectra have low noise, and accurately identify spectral features found in literature for test materials of organic photovoltaic crystals and transition-metal dichalcogenide thin films.