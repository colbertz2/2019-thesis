\subsection{Optoelectronic Materials}
\subsubsection{Organic Photovoltaics: ADT}
\subsubsection{Quantum Dots: \ce{CdSe}}
\subsubsection{Transition Metal Dichalcogenides: \ce{MoS2}}

\subsection{Photoluminescence}
Photoluminescence (PL) is a mechanism by which materials absorb and emit photons. The process can be described with respect to electronic transitions within an atom or molecule.

The absorptive transition occurs first, when a photon interacts with a molecule and is absorbed. The photon's energy must be approximately equal to the bandgap of the absorbing molecule to satisfy the energy transitions allowed by quantum mechanics. When that condition is met, the photon's energy raises an electron to an excited state, where it stays for a short time.

Some number of vibronic transitions occur as the electron loses energy to radiation and vibration. \todo{How do we measure these transitions, or their lifetimes?}

Finally, the radiative transition occurs when the electron decays back to a ground state. During the radiative transition, a photon is emitted at the bandgap energy as the electron moves from the lowest vibronic state in the conduction band to the highest vibronic state in the valence band. \todo{This needs some work.}

\subsection{PL as a Characteristic Measurement}
\todo{Why is PL a useful measurement in solid state? In biological sciences? In general?}