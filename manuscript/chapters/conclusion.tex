The goal of this project was to design and build a photoluminescence spectrometer with basic components, capable of exciting samples on the micron scale. We accomplished this using a simple diode laser system, metallurgical microscope, portable fiber-coupled spectrometer, and a few basic optical components. We demonstrated the functionality of the new instrument by measuring samples of photoluminescent materials ADT TES-F and cadmium selenide quantum dots, then compared our results to previous publications and our own measurements of the same samples taken with a Horiba Fluorolog-3 spectrophotometer.

This project handled the fundamental optical design and functionality of our new instrument; future students could expand on this work in a variety of ways. The microspectrometer can be fitted with different detectors for measuring PL at wavelengths outside the visible spectrum. New laser diodes can be used to excite samples at different wavelengths; a tunable laser system could be coupled into the instrument to enable PL excitation measurements. Optical power measurements --- of excitation or emission light --- would be a very useful feature to add to this instrument in the near future, and would enable more sophisticated characterization of optoelectronic materials.