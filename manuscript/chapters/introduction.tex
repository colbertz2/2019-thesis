% \section{The Need for a Simpler Spectrometer (Motivation)}
The Graham Micro-Femto Energetics ($\mu fE$) group at Oregon State University uses advanced optoelectronic methods to characterize materials, especially thin-layer materials and micron-scale semiconducting devices.

The group's workhorse when it comes to PL measurements is a Horiba \textbf{Something?} fluorimeter, coupled to a \textbf{Something?} microscope. The system uses a \textbf{Specs?} lamp and double-monochromator to illuminate a wide field with a tunable wavelength.

The wide field illumination is really nice for imaging, but makes it hard to isolate emissions from small spatial domains.

Something about visible and NIR detectors. NIR detector requires liquid nitrogen.

Something about the system takes a long time to warmup before the light source is stable.

Something about training time and the learning curve for using the device. Software is kinda complicated. Takes a really long time to take a measurement.

To address this, I build my system to be user friendly and easy to learn.

Naturally, my system has fewer applications. It only measures emission (not excitation), and modifications to the system are a little more complicated. But it's great for taking initial measurements of a sample. If more complicated stuff needs to be done, use the fluorimeter.

Also, my system has much better spatial resolution because it uses a laser source. This makes it less good for imaging, but that can be overcome.