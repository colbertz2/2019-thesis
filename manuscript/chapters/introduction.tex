Photoluminescence (PL) is the process by which a molecule re-emits a photon after having absorbed a photon of sufficient energy to excite an electron out of its ground state. Fluorescence is PL for which the lifetime of the electron in its excited state is short; on the order of nanoseconds. Fluorescence is commonly measured to characterize the band structure of molecules. Basic fluorescence measurements are emission and excitation; Emission measurements excite the sample at a fixed wavelength (typically in the ultraviolet range), and measure the relative intensity of light emitted over many wavelengths. Excitation measurements vary the wavelength used to excite the sample and measure the intensity of light emitted at a fixed wavelength. By combining these two methods, one can build an emission map--a surface which describes the sample's fluorescence as a function of both emission and excitation wavelengths. The scope of this project includes only emission measurements.

The Micro-Femto Energetics ($\mu fE$) group at Oregon State University uses advanced optoelectronic methods to characterize materials, especially thin-layer materials and micron-scale semiconducting devices. The group's workhorse when it comes to PL measurements is a Horiba Fluorolog-3 spectrofluorometer ("the fluorimeter"), often coupled to a microscope via optical fibers. The fluorimeter's excitation light source is a 450 W xenon arc lamp. A double-grating monochromator selects the chosen excitation wavelength, and this excitation light is directed into an optical fiber by a flat mirror. Delivering the excitation light to the sample via optical fiber provides flexibility when reconfiguring the system to meet the specific needs of a project. For example, the microscope can be used in a transmission or reflection mode by illuminating the sample from below or above, respectively. With the aid of computer software, the fluorimeter is able to measure both fluorescence emission and excitation.

While the Fluorolog-3 is an incredibly flexible and useful instrument, it is particularly limited with regards to how it illuminates samples. Using the intrument's sample chamber or while coupled to a microscope in transmission mode (such that light transmitted by optical fiber is incident directly on the sample), a region on the scale of centimeters is excited. Even in reflection mode, the excited region is made only proportionally smaller than the cross-sectional area of the optical fiber as the excitation light passes through the objective lens. For many applications, particularly biological fluorometry and fluorescence imaging, excitation on this scale is called for. For others, it may be desirable to excite highly localized, micron-scale regions of a sample.

With the goal of taking similar fluorescence measurements on a much smaller scale, we designed and built a visible photoluminescence microspectrometer. In its current form, the instrument is capable of measuring fluorescence emissions in the visible spectrum, using a diode laser as it's excitation source. This paper reports the process of designing the instrument, lessons learned during construction, and demonstrates the use of the instrument on two distinct samples of fluorescent molecules.
