% \section{The Need for a Simpler Spectrometer (Motivation)}

Photoluminescence (PL) is the process by which materials will absorb a photon, exciting an electron to a higher-energy "excited state," then emit a photon as the electron relaxes back to a lower-energy state. Measuring PL in different ways is a common method for characterizing semiconducting materials.

Basic PL measurements are emission and excitation, and differ by their independent variables. Emission measurements use one wavelength to excite the material, and measure the intensity of light emitted across a chosen spectrum. Excitation measurements use many wavelengths to excite the material, and measure the intensity of light emitted at a particular wavelength. This project is centered around PL emission measurements.

The Micro-Femto Energetics ($\mu fE$) group at Oregon State University uses advanced optoelectronic methods to characterize materials, especially thin-layer materials and micron-scale semiconducting devices.

The group's workhorse when it comes to PL measurements is a Horiba \textbf{Something?} fluorimeter, coupled to a \todo{model?} microscope. The instrument uses a \todo{specs?} xenon-arc lamp and double-monochromator to illuminate a wide field with a tunable wavelength. The light source is coupled to the microscope with a fiber optic cable. The instrument can be configured for reflection or transmission microscopy depending on the application. With the aid of computer software, the instrument is able to measure both PL emission and excitation. However, there are a few distinct challenges to using this instrument.

Because the fiber optic delivers a large beam, a large area on samples is illuminated whether the instrument is in a reflection or transmission mode. The wide-field illumination is excellent for imaging, but makes it hard to isolate emissions from small spatial domains.

\todo{Something about the system takes a long time to warmup before the light source is stable.}

\todo{Check specs for monochromator. What is it's spectrum like vs. laser diode? More or less monochromatic?}

\todo{Something about training time and the learning curve for using the device. Software is kinda complicated. Takes a really long time to take a measurement.}

This project offers a solution to these challenges by designing and assembling a new PL microspectrometer that can measure the PL emission of samples accurately, quickly, and with the ability to illuminate small spatial domains.

Naturally, the new instrument has fewer applications. It only measures emission (not excitation), and reconfiguration for other measurements requires more consideration to optical design. The goal of this project, however, is to design an instrument that is simple to use for PL measurement when use of the fluorimeter is impractical, too time consuming, or the instrument is in use for another experiment.
