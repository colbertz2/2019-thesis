\section{Photoluminescence}
Photoluminescence (PL) is a mechanism by which materials absorb and emit photons. The process can be described with respect to electronic transitions within an atom or molecule.

The absorptive transition occurs first, when a photon interacts with a molecule and is absorbed. The photon's energy must be approximately equal to the bandgap of the absorbing molecule to satisfy the energy transitions allowed by quantum mechanics. When that condition is met, the photon's energy raises an electron to an excited state, where it stays for a short time.

Some number of vibronic transitions occur as the electron loses energy to radiation and vibration. \textbf{How do we measure these transitions, or their lifetimes?}

Finally, the radiative transition occurs when the electron decays back to a ground state. During the radiative transition, a photon is emitted at the bandgap energy as the electron moves from the lowest vibronic state in the conduction band to the highest vibronic state in the valence band. \textbf{This needs some work.}

\section{PL as a Characteristic Measurement}
Why is PL a useful measurement in solid state? In biological sciences? In general?


\section{The Need for a Simpler Spectrometer}
The Graham Micro-Femto Energetics ($\mu fE$) group at Oregon State University uses advanced optoelectronic methods to characterize materials, especially thin-layer materials and micron-scale semiconducting devices.

The group's workhorse when it comes to PL measurements is a Horiba \textbf{Something?} fluorimeter, coupled to a \textbf{Something?} microscope. The system uses a \textbf{Specs?} lamp and double-monochromator to illuminate a wide field with a tunable wavelength.

The wide field illumination is really nice for imaging, but makes it hard to isolate emissions from small spatial domains.

Something about visible and NIR detectors. NIR detector requires liquid nitrogen.

Something about the system takes a long time to warmup before the light source is stable.

Something about training time and the learning curve for using the device. Software is kinda complicated. Takes a really long time to take a measurement.

To address this, I build my system to be user friendly and easy to learn.

Naturally, my system has fewer applications. It only measures emission (not excitation), and modifications to the system are a little more complicated. But it's great for taking initial measurements of a sample. If more complicated stuff needs to be done, use the fluorimeter.

Also, my system has much better spatial resolution because it uses a laser source. This makes it less good for imaging, but that can be overcome.


\section{Thin-Layer Materials}
\subsection{Anthradithiophene}
ADT TES-F came from Oksana.

Figure with the structure.

Why it's cool. Talk about OPVs in general. Potential applications.

It's been studied thoroughly by Oksana and past thesis students in Graham group. So it's a good calibration for the system, proof of concept.


\subsection{Cadmium Selenide}
This is literally something I had on hand. Find out more about why it's cool, structure, applications.


\subsection{Molybdenum Disulfide}
Get some samples from Ethan and measure them.

Talk about structure, application, find some data to compare to?
Talk about identifying monolayers using PL?